\documentclass{article}

\usepackage{amsmath, amsthm, amssymb}
\usepackage{geometry}
\geometry{margin = 3.0 cm}
\usepackage{float}
\usepackage{algorithm2e}
\usepackage{caption}
\usepackage[]{mcode}
\usepackage{algpseudocode}
\algblockdefx[NAME]{Input}{EndInput}
    [1][]{\textbf{Input:} #1}
    {}

\algblockdefx[NAME]{Output}{EndOutput}
    [1][]{\textbf{Output:} #1}
    {}

\newcommand{\myalg}[2]{\begin{figure}[h]
    { \renewcommand{\figurename}{Algorithm}
\begin{center}
\begin{minipage}{0.6\textwidth}
\hrulefill
\begin{algorithmic}[1]
    #1
\end{algorithmic}
\hrulefill
\end{minipage}
\end{center}
\caption{#2}
}
\end{figure}}

\renewcommand{\l}{\lambda}
\renewcommand{\L}{\Lambda}
\renewcommand{\mod}{~\mathrm{mod}~}
\renewcommand{\div}{~\mathrm{div}~} 
\renewcommand{\vec}[1]{\mathbf{#1}}

\renewcommand{\(}{\left(}
\renewcommand{\)}{\right)}

\newcommand{\dist}{\text{dist}}

\newcommand{\mat}[1]{\begin{pmatrix} #1 \end{pmatrix}}
\newcommand{\room}{\hspace{0.5cm}}
\newcommand{\hl}[1]{\textbf{#1}}
\newcommand{\ceil}[1]{\lceil #1 \rceil}
\newcommand{\floor}[1]{\lfloor #1 \rfloor}
\newcommand{\var}[1]{$\mathtt{#1}$}
\newcommand{\mc}[1]{\mathcal{#1}}
\newcommand{\hot}[1]{\mathcal{O}\(#1\)}
\newcommand{\minimx}[4]{\begin{pmatrix}{#1} & {#2} \\ {#3} & {#4} \end{pmatrix}}

\newcommand{\uin}{u_i^n}
\newcommand{\uinp}{u_i^{n+1}}
\newcommand{\uinm}{u_i^{n-1}}
\newcommand{\uipn}{u_{i+1}^n}
\newcommand{\uimn}{u_{i-1}^n}
\newcommand{\uimnp}{u_{i-1}^{n+1}}
\newcommand{\uipnm}{u_{i+1}^{n-1}}
\newcommand{\uipnp}{u_{i+1}^{n+1}}

\newcommand{\ujn}{u_j^n}
\newcommand{\ujnp}{u_j^{n+1}}
\newcommand{\ujnm}{u_j^{n-1}}
\newcommand{\ujpn}{u_{j+1}^n}
\newcommand{\ujmn}{u_{j-1}^n}
\newcommand{\ujmnp}{u_{j-1}^{n+1}}
\newcommand{\ujpnm}{u_{j+1}^{n-1}}
\newcommand{\ujpnp}{u_{j+1}^{n+1}}

\newcommand{\xin}[1]{{#1}_{i}^n}
\newcommand{\xinp}[1]{{#1}_{t i}^{n+1}}
\newcommand{\xtinm}[1]{{#1}_{t i}^{n-1}}
\newcommand{\xipn}[1]{{#1}_{t i+1}^n}
\newcommand{\ximn}[1]{{#1}_{t i-1}^n}
\newcommand{\ximnp}[1]{{#1}_{t i-1}^{n+1}}
\newcommand{\xipnm}[1]{{#1}_{t i+1}^{n-1}}
\newcommand{\xipnp}[1]{{#1}_{t i+1}^{n+1}}

\newcommand{\yin}[1]{{#1}_{i}^n}
\newcommand{\yinp}[1]{{#1}_{i}^{n+1}}
\newcommand{\ytinm}[1]{{#1}_{i}^{n-1}}
\newcommand{\yipn}[1]{{#1}_{i+1}^n}
\newcommand{\yimn}[1]{{#1}_{i-1}^n}
\newcommand{\yimnp}[1]{{#1}_{i-1}^{n+1}}
\newcommand{\yipnm}[1]{{#1}_{i+1}^{n-1}}
\newcommand{\yipnp}[1]{{#1}_{i+1}^{n+1}}

\newcommand{\vin}{v_i^n}
\newcommand{\vinp}{v_i^{n+1}}
\newcommand{\vinm}{v_i^{n-1}}
\newcommand{\vipn}{v_{i+1}^n}
\newcommand{\vimn}{v_{i-1}^n}
\newcommand{\vimnp}{v_{i-1}^{n+1}}
\newcommand{\vipnm}{v_{i+1}^{n-1}}
\newcommand{\vipnp}{v_{i+1}^{n+1}}

\newcommand{\dt}{\Delta t}
\newcommand{\dx}{\Delta x}
\newcommand{\dxi}{\Delta \xi}
\newcommand{\ctx}{\frac{c\dt}{\dx}}
\newcommand{\half}[1]{\frac{{#1}}{2}}




\newtheorem{lem}{Lemma}

\usepackage{tikz} \usetikzlibrary{shapes}
\usetikzlibrary{arrows}

\tikzset{node distance=3cm, auto}

\begin{document}
\title{Exercise F4a}
\author{Abe Wits (3629538)}
\date{2015/04/15}%TODO the date

\maketitle

\setlength{\parskip}{0.2 cm}
\setlength{\parindent}{0.0 cm}

\section*{Write models 1 and 2 as a system of two first-order PDEs (in time) as discussed in exercise \textbf{SC1}}
We can rewrite $u_{tt} = u_{xxx}$ as;
$$
\begin{cases}
v = u_t\\
v_t = u_{xxx}
\end{cases}
$$

\section*{Describe and implement a boundary-value (BV) technique for both models. For a description of BV we refer to the documents on the web page of this course. Note that different choices can be made for the time-integration part of BV and also for its (extra) final condition at $t=T$}
We have to make our PDEs discrete. We use leapfrog for our time integration and we use the $\hot{(\dx)^2}$ space discretization we derived in SC1;
$$u_t \approx \frac{\uinp - \uinm}{2\dt}$$
$$u_{xxx} \approx \frac{-\half{u_{i-2}^n}+\uimn-\uipn+\half{u_{i+2}^n}}{(\dx)^3}$$

We want to construct a system of linear equations that solves our PDE. The equations above can be used to construct linear constraints: we want to enforce that $v=u_t$ and $v_t=u_{xxx}$;
$$\uinp-2\dt\vin-\uinm = 0$$
$$\vinp-\frac{2\dt}{(\dx)^3}\(-\half{u_{i-2}^n}+\uimn-\uipn+\half{u_{i+2}^n}\)-\vinm = 0$$
In this system we see $\uin$ and $\vin$ as our variables. Let $T = N\dt$, then $n\in\{0,1,\dots,N\}$. Similarly we choose $x_M=x_{max}$ and $x_0=x_{min}$, with $m \in \{0,1,\dots,M\}$. We will handle boundary conditions in the space direction separately for the different models.
Notice that our constraints are invalid if they contain non-existent variables such as $u_i^{N+1}$, $v_i^{N+1}$, $u_i^{-1}$ or $v_i^{-1}$. We do not use such invalid constraints, but add boundary conditions for $n=0$ and $n=N$ instead:
$$u_i^0 = u_0(x_i)$$
$$u_i^N = u_T(x_i)$$
$$v_i^0 = v_0(x_i)$$
$$v_i^N = v_T(x_i)$$
The functions $u_0$, $v_0$ (the initial conditions)  and $u_T$, $v_T$ (the final conditions) are different functions for model 1 and model 2.
If all boundary conditions are chosen correctly, the number of constraints is exactly equal to the number of variables, hence the system can be solved numerically. We will write the system in the standard form, $A w = b$, where $A$ is a (known) matrix with all constraints, $b$ is a (known) vector with all the constants from the constraints, and $w$ the (unknown) vector that will contain our solution. We will handle the construction of $A$ and $b$ separately for model 1 and model 2.

\subsection*{Model 1}
We have periodic boundary conditions, so $x_0 = x_{min} = x_{max} = x_M$. Let $w=(w^0, w^1, \dots w^N)^T$, where $w^n=(w_0^n,w_1^n,\dots,w_{M-1}^n)$ (we don't need $w_M^n$ since it is equal to $w_0^n$) and $w_i^n = (\uin,\vin)$. If $w$ is implemented as a ``flat'' array, we have $\uin$ on (1-based) index $n2M+\hat i 2+1$ and $\vin$ on (1-based) index $n2M+\hat i 2+2$, where $\hat i = (i+M)\mod M$ (periodic boundary conditions).% The correctness of these formulas can be seen as follows; $u_0^0$ has index 1. $u_{i+1}^n$ has an index that is 
We will place our constraints on the rows of $A$


\end{document}

